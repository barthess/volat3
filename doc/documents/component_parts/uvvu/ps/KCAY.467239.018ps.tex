\newcommand{\rootdir}{../../../..}				% корень всех документов
\newcommand{\localtex}{\rootdir/tex}	        % файлы настроек
\newcommand{\templatedir}{\rootdir/templates}	% общие для всех шаблоны
 
\documentclass[
	14pt,
	russian,
	utf8,
        columnsxxiv, % первичное применение
        nocolumnxxxi,
	nocolumnxxxii, 
	pointsubsection,
	floatsingle,
	listlikepar,
	%footnoteasterisk
	]
{eskdtext}

%\special{papersize=a4}

\usepackage{cmap} % ��������� ������ ������� ���� � PDF (pdflatex), ��������� ����� ������!
% \usepackage[math]{pscyr} % PSCyr ��������� ����� inputenc � babel!!!
\usepackage{mathtext} % ��������� ��������� � �������������� ������. ��������� �� fontenc
\usepackage{amssymb,amsmath} %
% \usepackage{nccmath} % ��������� ����� amsmath


\usepackage[T2A]{fontenc}



%\usepackage{listings}
\usepackage{eskdlongtable}  %��� ������, ������� ������� ����� ��������
  % ������������ ����� ����� ������ ��� longtable.sty
  % ��� ������������ ���� 2.105
  \setcounter{LTchunksize}{100} %���������� �����, �������
  \setlength{\LTleft}{0pt}  % ������� ������� � ������ ����
%\usepackage[pageshow]{xtab} %�������������� ���������� ������� ������
  %��� �� ����������� ��-�� ������� ������� ����������� �� ����� �������
\usepackage{array}    %��������� ��� _����_ ������ ������� �����������
  %�������� �������������� �������� m{..} � b{..}
  %� ��� �� >{} � <{}
\usepackage{tabularx} %��� ������ � �������������� �������� ������ �������
  %\tracingtabularx %tabularx ����� � ������ ������ ���� ���������
\usepackage{tabulary} %��� ������ � ��� ����� �������������� �������� ������ �������
\usepackage{textcomp}   %��������� �������� ������ ���. �������,
  %� ��������� ������ ������� (\textcelsius)
\usepackage{xcolor}   %�� �������� �������
\usepackage{multirow} %��������� ���������� ������ ������� �� ���������
\usepackage{graphicx} %��� ������� ��������
  \graphicspath{{\rootdir/common_images/}{./include/}} %������ ������� ���
  %\graphicspath{{../../../../common_images/}{../../common_images/}{./include/}} %������ ������� ���
\usepackage{afterpage}  %��� ������� ������� �������� �� ��������� ��������
%\usepackage{multicol}
\usepackage{pdflscape}  %��� ������ � ��������� ����������,
  %��� ���� ���������������� �� ������ �����, �� � ���
  %��������, ����� �� ������� ������� ��� ���������
  %�������� ������ � pdf
%\usepackage{layout}    %��������� ����������� ��������
  %\layout ���������� ������������ �� �������� �������
\usepackage{calc}   %��������� ����������� ������������ ������� ����������
% \usepackage{flafter} %��������� � ���, ����� ��� �������� ���� �������� ����� ������ �� ���
%\usepackage{colortbl}  %������� �������

%\usepackage{showkeys}  %��� ������� ������� � ��������� �� ��������
  %������������� ��������� ������ �� ����� �������

%\usepackage[bookmarksopen=true,bookmarksnumbered=true,colorlinks=true,unicode]{hyperref}
  %���� ����� ����� ��� "������������" ������.
  %�� ������ ������������ ���������
  %�� � ��� ������-�� � ����� ����������� ������
  %������ �����



\usepackage{eskdchngsheet}  %���� ����������� ���������

	% общие для всех документов пакеты
\hyphenation{
БИУС
ГОСТ
ГЛОНАСС
АНРЕЛ
АВДАТ
РЕПЛИКАТОР
ис-пол-ни-тель-ных
При-ме-ча-ние
при-ме-ча-ние
Мне-мо-ни-чес-кое
}
	% список сллов, которые нельзя разрывать переносами


% Cамодельные пакеты 
\usepackage{\localtex/commondata}
\usepackage{\localtex/macros}
\usepackage{\localtex/layout}
\usepackage{\localtex/kamertonstyle}
\usepackage{\localtex/passportkamertonstyle}
\usepackage{\templatedir/passport/passport_macros}	% паспортспецифичные макросы

%========================================================================
%======== переменные для автозаполнения паспорта ========================
%========================================================================
% для облегчения создания паспортов на 
% другие устроЙства копипастом
\newcommand{\nameDevice}{\nameUVVU}
\newcommand{\nameDeviceAbbr}{\abbrUVVU}
\newcommand{\nameDeviceFull}{\nameUVVU{} \abbrUVVU}
\newcommand{\ksauDevice}{\ksauUVVU}


% названия устройства в падежах, отличных от именителного
\newcommand{\genitivecasename}{\abbrUVVU}         % родительном падеже
\newcommand{\prepositionalcasename}{\abbrUVVU}    % предложный падеж

% род устройства (мужской/женский), нужен дальше по тексту
% для правильного использования зависимых 
% слов (напр. <<вышедшей из строя>>).
% Принимает значения либо female, либо что угодно другое, в т.ч. пустую строку
\newcommand{\gender}{}


%========================================================================
%======== заполнение основной надписи ===================================
%========================================================================
\ESKDclassCode{\markout{ОКП 68 1100}}
\ESKDtitle{\nameDevice}
\ESKDdocName{Паспорт}
\newcommand{\ESKDtheDocNameAbbr}{ПC}
\ESKDsignature{\ksauDevice[\ESKDtheDocNameAbbr]}
\ESKDauthor{Пылинский}
\ESKDchecker{Артемьев}
\ESKDnormContr{Степаненко}
\ESKDdate{2009/12/11}
\ESKDcolumnXXV{\ksauDevice}

%========================================================================
%======== собсна документ ===============================================
%========================================================================
\begin{document}
\makekamertontitle
% \maketitle

\section{�������� �������� �� ������� � ����������� ������}
  \subsection{�������� �������� �� �������}
  
  %��������������
  %подключается в файле /templates/passport/specs

\point {\nameDevice} ({\nameDeviceAbbr}) предназначен:
  \begin{itemize}
    \item для сбора информации с дискретных, аналоговых и частотных
      датчиков шасси, ее предварительной обработки и передачи в 
      вычислитель бортовой ВБ;
    \item для управления исполнительными механизмами шасси.
  \end{itemize}

  
    \point {\nameDevice} �������������  ������  1.7  ����������  �  ��
      \gost{� 20.39.304-76} � �����������, ������������ 
      �~\ref{pt:impact_factors}.
    \point ������������ "--- \nameDevice.
    \point ����������� "--- \ksauDevice.
    \point ������������ "--- \nameManufacturer.\par\bigskip
    \begin{tabular}{@{}ll}
      \multicolumn{2}{@{}l}{\addressManufacturer}\\
      ���./����:  &(0-17) 200-57-98\\
      ���.:   &(0-17) 222-17-89\\
      e-mail :  &kamerton@kamerton.by\\
    \end{tabular}
    \par\bigskip
    \makebox[0.5\textwidth]{��������� ����� \hrulefill}\par 
    \makebox[0.5\textwidth]{���� ������������ \hrulefill}



\newpage
  \subsection{����������� ��������������}
    \point �������� ����������� ������ {\genitivecasename} ��������� � 
    �������~\ref{tab:replicator_specs}.

    \begin{table}[h]
      \caption{�������� ����������� �������������� {\genitivecasename}}
      \label{tab:replicator_specs}
      % ������� ������ ����������� ���������� � ��� ������

% ������ ��� ���������� ������������� �������
% Tут собраны для основные технические спецификации
% устройства. 
% В отдельный файл они были вынесены исключительно
% для удобства включения в другие документы (например,
% руководство по эксплуатации)
 
\renewcommand{\nominalsupplypower}{20} % потребляемая мощность, ватты



\begin{tabularx}{\textwidth}{|X|p{3.1cm}|p{3.1cm}|}
  \hline
  %--- �����---------------------------------
  \multicolumn{1}{|c|}{������������ ���������}&
  \multicolumn{1}{c|}{��������}&
  \multicolumn{1}{c|}{����������}\\\hline 
  
  %-----����������---------------------------
  \setcounter{tablerowcounter}{0}
  \inTableEnum ���������� ���������� �������� ����, �   &��~{\minsupplyvoltage}
                              ��~\maxsupplyvoltage &\\\hline
  \inTableEnum ������������ ��������, ��    &\nominalsupplypower, �� �����  &\\\hline
  \inTableEnum ����� ����������� ������, �����  &\continuousworkingtime, �� ����� &\\\hline
  
  \inTableEnum ���������� �������, ��&
  \dimensionsReplicator, \mbox{�� �����}&\\\hline

  \inTableEnum �����, ��, �� �����&\massReplicator & \\\hline

\end{tabularx}

    \end{table}


\newpage
    \point �������� ������� �������������� �������� ��������� 
      � �������~\ref{tab:env_impact}.
      \label{pt:impact_factors}

    {
    \par
    \setcounter{tablerowcounter}{0} %����� �������� ������� (��� ������������ ������� \inTableEnum)
    \begin{table}[h!]
      \caption{��������� ���������� �������� ������� �������������� ��������}
      \label{tab:env_impact}
      %\input{\rootdir/common_data} % ������ �� ������� ������������

\begin{tabular}{|b{0.7\textwidth}|b{0.25\textwidth}|}
  \hline
  % �����===================================================================
  \multicolumn{1}{|c|}{������������ �������� ��������������� �������} &
    \multicolumn{1}{c|}{��������}
      \\\hline
  % ����������==============================================================
  \setcounter{tablerowcounter}{0}
  
  \inTableEnum ����������� ���������� �����, \textcelsius:          &\\
    \inTableItemize ���������� ����������                 &\temperaturestoragemax, �� ����� \\
    \inTableItemize ������� ����������                    &\temperatureworkingmax, �� ����� \\
    \inTableItemize ������� ���������� ��� ������ ���������� �� �� � ��� &\mbox{����� 30}, �� �����\\
    \inTableItemize ������� ���������� ��� ������ ���������� �� ���      &\temperaturestoragemin, �� �����\\
    \inTableItemize ���������� ����������                 &\temperaturestoragemin, �� �����\\\hline
    
    \inTableEnum ���������� ������������� ��������� 
    ������� ��� ����������� \airhumiditywithtemp{}
    \textcelsius, \%                                  &\ \newline\airhumidity, �� �����\\\hline
        
  \inTableEnum ����������� ���������� 
        ��������, \mbox{��� (�� ��. ��.):}                &\\
    \inTableItemize �������                                         &\pressureworkingmin, �� �����\\
    \inTableItemize ��� ����������������� �� ������ 10000",�         &\pressurestoragemin, �� �����\\\hline
    
  \inTableEnum �������������� ��������:                             &\\
    \inTableItemize ���������, �/�\texttwosuperior\ (g)   &\vibrationacceleration, �� �����\\
    \inTableItemize �������� ������, ��               
            &��~{\vibrationfrequencymin} ��~\vibrationfrequencymax\\\hline
    
  \inTableEnum ������������ ����� ���������� ��������:&\\
    \inTableItemize ������� �������� �������� ���������, 
            �/�\texttwosuperior\ (g)        &\shockacceleration, �� �����\\
    \inTableItemize ������������ �������� ��������, ��
            &��~{\shockpulsewidthmin} ��~\shockpulsewidthmax \\\hline
\end{tabular}

    \end{table}
    \par
    }

  \subsection{���������� ���������}
    ������� ��������� �� ����� \genitivecasename "--- \MTBF~�. 
    �������� ������ "--- �������������� �������� ���������� 
    ����������� ���������������, ����������� � 
    ��������� \MakeLowercase{\ESKDtheDocName}�.

  \subsection{�������� � ���������� ����������� ���������� � �������
    ��������}
    \point � {\prepositionalcasename} ���������� ��������� ����������� 
      ���������\footnote{�������� � ���������� ����������� ����������
      ��������� �����������.}:

    \begin{itemize}
      \item \makebox[0.3\textwidth]{������ "--- \hfill �};
      \item \makebox[0.3\textwidth]{������� "--- \hfill �}.
    \end{itemize}

    \point � {\prepositionalcasename} ���������� �������� � �����������
      \mbox{������� "---\qquad\qquad ��}.
               % технические характеристики
% ������ ���� ������������� ������ ����� �������
% ���������� ���������� �������� � ���� completeness_content.tex,
% ������� ������ ���������� ����� � ������� ������
% ��������/��������� 
% ��������� ��� ����, ��� ������� ������� �� 4 �������
\newpage

\section{�������������}

{

  \begin{table}[h]
    \caption{������������� �������� \genitivecasename}
    \centering
    % \small
    \begin{tabularx}{\textwidth}{|m{7.2cm}|c|l|X|}
      \hline
      ������������, �����������&����������&��������� �����&����������\\\hline
      \nameDevice\newline \ksauDevice&1&&\\\hline
\ESKDtheDocName\newline \ksauDevice[\ESKDtheDocNameAbbr]&1 &&\\\hline
Упаковка& 1 & & Подборная \newline упаковка\newline изготовителя\\\hline

    \end{tabularx}
    \label{tab:replicator_complect}
  \end{table}
}
        % комплектоность
\newpage

% ��� ������������� 14 ������ ���� ����� ������ ����-����
% �� ���������� �� ��������. �������� ����������� ��������
% ��� ������ �������� � 1.25 �� 1.20
{
\makeatletter
\ifthenelse{\equal{\ESKD@docfont}{14pt}} 
      {\renewcommand{\baselinestretch}{1.20}\selectfont}
      {} 
\makeatother

\section{�������, ����� ������ � ��������, �������� ������������
  (����������)}
  \subsection{�������, ����� ������ � ��������}
    ������  �� ������� ������������ ������� "--- ���� 
    ��� (��� ������� ��������� �� ����� \MTBF~�) �
    ������� �������� ����� ������ 20 ���, � ��� ����� 
    ���� �������� ���� ��� � �������� ������������ � 
    �������� �������������� ����������.

    ������� ���� ������������� � �������� ������������ "---
    10 ��� � ���������������� ����� ���� ���.

    ��������� �������, ���� ������ � ���� �������� 
    ������������� ��� ���������� ������������ ����������, 
    ������������� ���������������� �������������.
    

  \subsection{�������� ������������ (����������)}
    ������������ ����������� ������������ �������� {\genitivecasename}
    ����������� ����������� ����������� ������������ ��� 
    ���������� ������������ ������ ������������, �������� � 
    �����������������, ������������� ���������������� �������������.

    ����������� ���� �������� ��������� � ����������� ������ ������������ 
    10 ��� �� ��� ����� � ������������, �� �� ����� 11 ��� �� ��� ������������.

    �������������� {\genitivecasename},
    \ifthenelse{\equal{\gender}{female}}
    {��������}
    {���������}
    �� ����� � ������� 
    ������������ �����, �������������� �� ���� ������������. 
    ��� �������������� (����� �����) � ����������� ������, ���� 
    �������� �� 
    \ifthenelse{\equal{\nameDevice}{\nameDeviceAbbr}}
    {\nameDeviceAbbr}
    {\MakeLowercase{\nameDevice}}
    ������������ �� ����� ��������������.

    ����������� ���������� ����� �� ����������� ������������ ���:
    \begin{itemize}
      \item ��������� ������� ������������ � ��������;
      \item ����� {%
        \ifthenelse{\(\equal{\ESKDtheDocName}{�������}\)\or\(\equal{\ESKDtheDocName}{��������}\)}
        {\MakeLowercase{\ESKDtheDocName}�}
        {����������� �� ������������};}
      \item ��������� �����;
      \item ������������������� ���������� ����������� � ����� {\genitivecasename}.
    \end{itemize}
}
            % гарантия и сроки службы
\newpage
\makeatletter
\section{�����������}


\cI=1.6cm
\cII=6cm
\cIII=2.5cm
\cIV=6cm

  \begin{table}[!ht]
    \centering
    \caption{����������� \genitivecasename}
    \begin{tabular}{|p{\cI}|p{\cII}|>{\centering}m{\cIII}|m{\cIV}|}
      \hline
      \multicolumn{1}{|c|}{����}&
      \multicolumn{1}{m{\cII}|}{\centering ������������ ��������}&
      \multicolumn{1}{m{\cIII}|}{\centering ���� ��������, ����}&
      \multicolumn{1}{m{\cIV}|}{\centering ���������, ������� � �������}\tabularnewline\hline
      {\mbox{}
      \protect\ifthenelse{\equal{\ESKD@docfont}{12pt}} 
      {\vspace{0.75\textheight}}
      {\vspace{0.72\textheight}}
      \mbox{}}&
        ����������� ���������� � ��������� ����� � 
        ����������-����������� \gost{3956-76}, � ������������&
          ����&\\\hline
    \end{tabular}
    \label{tab:replicator_conservation}
  \end{table}

\makeatother
\clearpage
        % консервация
\newpage

\section{������������� �� ������������}

\noindent
\nameksaunumber
\bigskip\noindent%
%
\blankfieldtxt[c]{\textwidth}
  {\centering\nameManufacturer}
  {������������ ��� ��� ������������}
  {��������}
  {}
  \par
%
\bigskip\noindent �������� �����������, ��������������� � ����������� ����������� ������������.

\bigskip
\noindent%
  \blankfieldtxt{0.3\textwidth}{}{���������}{}{} \hfill
  \blankfieldtxt{0.3\textwidth}{}{������ �������}{}{} \hfill
  \blankfieldtxt{0.3\textwidth}{}{�����������}{}{}

\bigskip
\noindent
  \blankfieldtxt{0.5\textwidth}{}{���, �����, �����}{}{}
 % свид. об упаковке
\newpage

\section{������������� � �������}

\noindent%
\nameksaunumber
\bigskip
  \noindent ���������� � ������ � ������������ � ������������� ������������ 
  ��������������� (������������) ����������, ����������� ����������� 
  ������������� � ������� ������ ��� ������������.\par
\bigskip

\begin{centering} 
  \ \par ��������� ��� \par
\end{centering}
\bigskip
\MPscript
\vfill
%\vspace{5cm}
\noindent \parbox{0.3\textwidth}{������������\\ �����������}
\hfill \blankfieldtxt[t]{0.5\textwidth}{\centering\ksauDevice[��]}{
  ����������� ���������,\par �� �������� ������������ ��������}{}{}

\bigskip
\MPscript
    % свид. о приемке
\newpage

\section{Ремонт}

\subsection{Краткие записи о произведенном ремонте}

\noindent
\nameksaunumber
\bigskip
\noindent
\blankfieldtxt{0.7\textwidth}{}{предприятие; дата}{}{}

\bigskip
\noindent
Наработка с начала\\
\blankfieldtxt[c]{\textwidth}{}
  {параметр, характеризующий ресурс или срок службы}{эксплуатации}{}

\bigskip
\noindent
Наработка после последнего\\ 
\blankfieldtxt[c]{\textwidth}{}
  {параметр, характеризующий ресурс или срок службы}{ремонта}{}

% \vspace{5ex}
\noindent
\blankfieldtxt[c]{\textwidth}{}
  {}{Причина поступления в ремонт}{}

\noindent
\blankfieldtxt[c]{\textwidth}{}
  {}{}{}

% \vspace{5ex}%
\noindent%
\blankfieldtxt[c]{\textwidth}{}
  {вид ремонта и краткие}{Сведения о произведенном ремонте}{}
\noindent
\blankfieldtxt[c]{\textwidth}{}
  {сведения о ремонте}{}{}

\noindent\blankfieldtxt[c]{\textwidth}{}
  {}{}{}

\newpage

\subsection{Свидетельство о приемке и гарантии}
\noindent
\nameksaunumber

\noindent
\blankfieldtxt[c]{0.3\textwidth}{}
  {вид ремонта}{}{}\hfill 
\blankfieldtxt[c]{0.3\textwidth}{}
  {наименование предприятия, \\ условное обозначение}{}{}\hfill
\blankfieldtxt[c]{0.35\textwidth}{}
  {вид документа}{согласно}{}

\bigskip
\noindent Принят в соответствии с обязательными требованиями 
государственных (национальных) стандартов и действующей технической 
документацией и признан годным для эксплуатации.

\noindent
\blankfieldtxt[с]{\textwidth}{}
  {параметр, определяющий}{Ресурс до очередного ремонта}{}

\noindent
\blankfieldtxt[с]{0.85\textwidth}{}
  {ресурс}
  {}
  {в течение срока службы}
\hfill
\blankfieldtxt[с]{0.15\textwidth}{}
{}{}{лет}

\noindent
\blankfieldtxt[с]{\textwidth}{}
  {условия хранения лет (года)}
  {(года), в том числе срок хранения}
  {}

\bigskip{}\bigskip
Исполнитель ремонта гарантирует соответствие 
изделия требованиям действующей технической 
документации при соблюдении потребителем требований 
действующей эксплуатационной документации.

\begin{centering} 
  \ \par\bigskip Начальник ОТК \par
\end{centering}
\bigskip
\MPscript
	
\newpage

\section{������� �� ������������ � ��������}

	\subsection{������������}
		\point ������������ {\genitivecasename} ������������ � ������������ 
			� ������������ ��������� <<\nameMain. ����������� �� ������������
			\ksauMain[��]>>.

	\subsection{��������}
		\point {\nameDevice} � �������� ������������ ������ 
			��������� � �������� �������������� ��������� 
			� ������������ ����������� ������� ��� ����������� 
			��~{\temperaturestoragemin} 
			��~\temperaturestoragemax~\textcelsius\ � ������������� 
			��������� ������� �� ����� 98\% ��� 
			�����������~\airhumiditywithtemp~\textcelsius\ ��� ����������� �����.       % хранение использование
\newpage

\section{Сведения об утилизации}
	Утилизацию {\genitivecasename} проводит потребитель 
	по действующим у него нормативным документам 
	и положениям.

	Элементы, содержащие драгоценные материалы, 
	при демонтаже {\genitivecasename} подлежат учету, 
	хранению и сдаче в установленном порядке.



%========================================================================
%=========================== лист регистрации изменений =================
%========================================================================
\begin{ESKDchangeSheet}
  \ESKDchangeSheetFill
\end{ESKDchangeSheet}


\end{document}
% vim:tw=70
