%\special{papersize=a4}

\usepackage{cmap} % Поддержка поиска русских слов в PDF (pdflatex), ЗАГРУЖАТЬ САМЫМ ПЕРВЫМ!
% \usepackage[math]{pscyr} % PSCyr ЗАГРУЖАТЬ ПЕРЕД inputenc и babel!!!
\usepackage{mathtext} % Поддержка кириллицы в математическом режиме. ЗАГРУЖАТЬ ДО fontenc
\usepackage{amssymb,amsmath} %
% \usepackage{nccmath} % расширяет пакет amsmath


\usepackage[T2A]{fontenc}



%\usepackage{listings}
\usepackage{eskdlongtable}  %для таблиц, которые длиннее одной страницы
  % представляет собой набор патчей для longtable.sty
  % для соответствия ГОСТ 2.105
  \setcounter{LTchunksize}{100} %количество строк, которые
  \setlength{\LTleft}{0pt}  % прижать таблицы к левому краю
%\usepackage[pageshow]{xtab} %альтернативная реализация длинных таблиц
  %мне не понравилась из-за слишком больших промежутков на месте разрыва
\usepackage{array}    %добавляет для _всех_ таблиц вкусную возможность
  %задавать характеристики столбцов m{..} и b{..}
  %а так же >{} и <{}
\usepackage{tabularx} %для таблиц с автоматическим подбором ширины колонок
  %\tracingtabularx %tabularx будет в выводе писать свои сообщения
\usepackage{tabulary} %для таблиц с еще более автоматическим подбором ширины колонок
\usepackage{textcomp}   %позволяет печатать разные доп. символы,
  %в частности градус Цельсия (\textcelsius)
\usepackage{xcolor}   %по названию понятно
\usepackage{multirow} %позволяет объединять клетки таблицы по вертикали
\usepackage{graphicx} %для вставки рисунков
  \graphicspath{{\rootdir/common_images/}{./include/}} %искать рисунки тут
  %\graphicspath{{../../../../common_images/}{../../common_images/}{./include/}} %искать рисунки тут
\usepackage{afterpage}  %для вставки больших рисунков по окончании страницы
%\usepackage{multicol}
\usepackage{pdflscape}  %для листов в альбомной ориентации,
  %при этом переворачивается не только текст, но и вся
  %страница, чтобы не вертеть головой при просмотре
  %работает только в pdf
%\usepackage{layout}    %добавляет возможность командой
  %\layout посмотреть используемые на странице размеры
\usepackage{calc}   %добавляет возможность использовать простую арифметику
% \usepackage{flafter} %заботится о том, чтобы вся плавучка была всталена после ссылки на нее
%\usepackage{colortbl}  %цветные таблицы

%\usepackage{showkeys}  %над ссылкой печатет в документе ее название
  %целесообразно применять только во время отладки

%\usepackage[bookmarksopen=true,bookmarksnumbered=true,colorlinks=true,unicode]{hyperref}
  %этот пакет нужен для "кликабельных" ссылок.
  %он должен подключаться последним
  %но с ним почему-то в самых неожиданных местах
  %рвется текст



\usepackage{eskdchngsheet}  %лист регистрации изменений

